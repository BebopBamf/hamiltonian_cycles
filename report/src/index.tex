\section{Abstract}

Let us write a nice intro here.

\section{The Hamiltonian Cycle Problem}

\subsection{Hamiltonian Paths and Hamiltonian Cycles}

The hamiltonian cycle is similar to that of the hamiltonian path where inside an
undirected or directed graph, the hamiltonian path, also known as the traceable
path, is a path that visits each vertex exactly once. However, differing to the
hamiltonian path, the starting point and ending point of the path must also be
adjacent to each other such that they are able to create a cycle using an
available edge. 

\subsection{Mathematical Definition}

Here we define a mathematically rigerous defintion\cite{Ros07}. This defintion
is simply to add rigour to the defintion in order to construct mathematically
sound proofs.

\begin{definition}[Graph]
    A graph $G$ is a tuple with a set of verticies $V$ and edges $E$, with an
    edge connecting two verticies. Notated as $G = (V, E)$
\end{definition}

\begin{definition}[Simple Cycle]
    A simple cycle is a path $x_0, x_1, ..., x_{n-1}, x_n$ of a graph
    $G = (V, E)$ such that 
\end{definition}


\begin{equation}
    x_0, x_1, ..., x_{n-1}, x_n
\end{equation}

\subsection{The Hamiltonian Cycle Problem}

The hamiltonian cycle problem can be expressed in many ways depending on whether
you represent it as a decision problem (does a graph contain a hamiltonian
cycle, yes or no?) or whether the algorithm should return the actual path.

The hamiltonian cycle problem can be generally understood as does a graph
contain a hamiltonian cycle. It can also be understood as a general case of the
Travelling Salesman Problem, and algorithms that apply to the travelling
salesman problem apply to the hamiltonian cycle problem. However the inverse is
\emph{not} true.

\printbibliography

